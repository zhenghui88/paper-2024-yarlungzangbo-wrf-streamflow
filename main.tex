\documentclass[draft]{agujournal2019}
\usepackage{url}
\usepackage{lineno}
\usepackage[inline]{trackchanges}
\usepackage{soul}
\graphicspath{ {./fig/} }
% \graphicspath{ {.} }
\linenumbers{}

% \draftfalse

\journalname{Water Resources Research}

\begin{document}

\title{Evaluation of atmospheric models using parsimonious network routing model and streamflow observations over mountainous regions: A case study of the Yarlung Zangbo River on the Tibetan Plateau}

\authors{Heng Yang\affil{1}, Shuanglong Chen\affil{2}, Qingyun Bian\affil{3}, and Hui Zheng\affil{3}}

\affiliation{1}{Science and Technology Research Institute, China Three Gorges Corporation, Beijing 100038, China}
\affiliation{2}{Baihetan Hydropower Plant, China Three Gorges Corporation, Sichuan 615421, China}
\affiliation{3}{Institute of Atmospheric Physics, Chinese Academy of Sciences, Beijing 100029, China}

\correspondingauthor{Hui Zheng}{\url{hzheng_iap@outlook.com}}

\begin{keypoints}
\item A network routing-based method is proposed for atmospheric model evaluation, leveraging streamflow's superior spatial representativeness
\item Parsimonious routing method shows robustness in mountain basins, as celerity calibration is insensitive to the runoff source
\item Uncertainty of parsimonious routing model in mountainous region is sufficiently small to discriminate among atmospheric model configurations
\end{keypoints}

\begin{abstract}
  Evaluating kilometer-scale atmospheric models in data-sparse mountains is challenging because in-situ meteorological observations are scarce and remote-sensing products are uncertain. Using hydrological models to link atmospheric outputs to streamflow is equally problematic, as those models carry large structural and parameter uncertainties in mountain terrain. We therefore propose a simple, low-uncertainty alternative: route the runoff generated by the atmospheric model through a network routing model whose parameters are calibrated against observed streamflow. We apply this approach to assess thirteen 3-kilometer Weather Research and Forecasting (WRF) model experiments over the Yarlung Zangbo River basin on the Tibetan Plateau, configured with systematically varied parameterizations for radiation, microphysics, planetary boundary layer, and orographic drag. In mountainous basins, routing can be substantially simplified by applying the Muskingum model. This parsimonious model proves effective and robust: hourly streamflow simulations achieve a median Pearson correlation of 0.75 across all examined gauging stations, and the calibrated wave celerity shows little sensitivity to the driving WRF experiment. Statistical analysis confirms that routing-model uncertainty is small enough to distinguish performance differences among the WRF configurations. Compared with precipitation-based evaluation, the routing-based approach provides complementary and more hydrologically relevant measures of atmospheric model performance, especially when basin-mean precipitation is already realistic. The method offers a discriminative tool that leverages the superior spatial representativeness of streamflow observations to evaluate atmospheric models in data-sparse mountainous regions.
\end{abstract}

\section{Introduction}

Atmospheric models are essential for hydrological forecasting~\cite{najafi2024NC, cosgrove2024JAWRA, thielen2009HESS} and hydroclimatic projections~\cite{francois2019JH, wagner2016WRR, fowler2007IJC, maraun2010RG, dougherty2020JHM}. Recent advances have steadily increased the spatial resolution of atmospheric models to five kilometers or less~\cite{clark2016MA, prein2015RG, stevens2019PEPS, tang2023GMD, li2024ESSD}. These high-resolution simulations explicitly resolve deep convection, thereby obviating the parameterization that introduces substantial uncertainty in coarser-resolution models. Consequently, they are designated convection-permitting models~\cite{prein2015RG, clark2016MA, mooney2017JC, lucas-picher2021WCC}. Kilometer-scale simulations also capture fine-scale topographic influences on atmospheric circulation~\cite{lin2018CD, zhou2021CD, yuan2023AR, sugimoto2021JHM, ma2023CD, li2022AAS}. The explicit representation of deep convection and detailed topography enhances water cycle simulation accuracy~\cite{jiang2022HESS, sugimoto2021JHM, ma2023CD}. Evidence indicates that kilometer-scale atmospheric models can outperform in-situ observations~\cite{lundquist2019BAMS} and satellite-based products~\cite{jiang2022IJC} in mountainous regions, making their integration into hydrological applications particularly valuable~\cite{xie2025JH, zhang2021WRR, reszler2018NHESS, rudisill2023GMD}.

However, kilometer-scale atmospheric models do not always perform optimally. Achieving skillful simulations requires careful configuration of modeling domains, initial and boundary conditions, and process parameterizations~\cite{bauer2015N, prein2015RG, lucas-picher2021WCC}. Atmospheric model intercomparison projects, including the Atmospheric Model Intercomparison Project (AMIP)~\cite{gates1999BAMS}, the International Grand Global Ensemble (TIGGE) of the Observing System Research and Predictability Experiment~\cite{swinbank2016BAMS, parsons2017BAMS}, and the Coordinated Regional Climate Downscaling Experiment (CORDEX)~\cite{giorgi2009WMOB}, represent sustained efforts to systematically compare atmospheric models across diverse scientific objectives. In the Tibetan Plateau, a prominent mountainous region, the CORDEX Convection-Permitting Third Pole (CPTP) project~\cite{prein2023CD} intercompares kilometer-scale atmospheric models through numerous simulations across three selected periods, each lasting several days~\cite{prein2023CD}. These projects yield critical insights into atmospheric model uncertainties and performance variations~\cite{gates1999BAMS, parsons2017BAMS, giorgi2009WMOB, prein2015RG, prein2023CD}. The optimal configurations identified through these comparisons guide applications in climate projections, weather predictions, and regional downscaling.

Evaluation of atmospheric models is fundamental to these intercomparison efforts~\cite{prein2015RG, lucas-picher2021WCC}, yet remains particularly challenging in data-scarce mountainous regions. Current evaluation approaches predominantly depend on in-situ meteorological observations and remote-sensing products~\cite{prein2023CD, collier2024CD, kukulies2023JC, zou2024AAS, ma2023CD, zhou2021CD, karki2017ESD}. In mountainous terrain, however, in-situ observations are typically sparse and unevenly distributed, introducing substantial biases~\cite{miao2024PNAS, lundquist2019BAMS}. Remote-sensing products, including satellite-based precipitation and snow water equivalent estimates, provide enhanced spatial coverage but are subject to considerable uncertainties arising from complex terrain effects and sensor limitations~\cite{henn2018JH, behrangi2014JAMC, bian2019JHM}. The combination of sparse in-situ observations and uncertainties in remote-sensing data compromises the reliability of kilometer-scale atmospheric model assessments. \citeA{collier2024CD} conducted the first ensemble of kilometer-scale atmospheric simulations spanning a complete hydrological year (October 2019 to September 2020) over the Tibetan Plateau, finding that substantial discrepancies among observational products impeded model evaluation. This challenge is not unique; the CORDEX-CPTP project similarly considers the identification of suitable observational data for atmospheric model evaluation as a critical priority~\cite{prein2023CD}.

Streamflow observations constitute a promising basis for evaluating atmospheric models. As the integrated response of a catchment to meteorological variables such as precipitation, temperature, and evapotranspiration, streamflow encapsulates the cumulative effects of these variables across temporal and spatial dimensions~\cite{beven2006HESS, henn2015WRR, henn2016WRR}. This characteristic provides an inherent advantage over in-situ meteorological observations in spatial representativeness. Additionally, as direct in-situ measurements, streamflow observations are generally more reliable than remote-sensing products.

Several methodologies have been developed to evaluate atmospheric models using streamflow observations. The most direct approach compares atmospheric model-simulated runoff directly with streamflow observations~\cite{jiang2022IJC}. This method assumes that the routing process from runoff generation to streamflow gauges is negligible. This assumption is valid only for small basins or longer time scales where flow wave travel time constitutes a negligible fraction of the analysis period. \citeA{chen2025W} conducted control experiments with and without routing processes in the Upper Jinsha River basin at the eastern edge of the Tibetan Plateau. They demonstrated that routing effects can be neglected at monthly time scales but significantly influence streamflow variations at daily and sub-daily resolutions. These findings align with \citeA{allen2018GRL}, who reported typical flow wave travel times on the order of days. Consequently, direct runoff--streamflow comparison is primarily suitable for headwater catchments or evaluations at monthly and longer time scales. Applying this method, \citeA{jiang2022IJC} evaluated basin-wide, multi-year average runoff from atmospheric models. However, this approach is inadequate for daily or sub-daily evaluations or for medium-to-large watersheds.

Another common approach uses hydrological models to evaluate atmospheric models~\cite{pang2020HESS, zhang2022CD, xie2025JH}. This methodology uses precipitation and other surface meteorological variables simulated from atmospheric models to drive hydrological models, followed by comparison of simulated and observed streamflow~\cite{krier2012WRR, pang2020HESS}. However, hydrological models are built upon numerous scientific assumptions about model structure and parameters~\cite{clark2011WRR, kirchner2009WRR, henn2016WRR}. While these assumptions can be adequately tested in data-rich regions~\cite{clark2011WRR, zheng2020JAMES}, they remain largely unconstrained in data-sparse mountainous areas such as the Tibetan Plateau. \citeA{lei2025JH} investigated streamflow uncertainties stemming from meteorological inputs and hydrological model structures in the Yarlung Zangbo River basin at the southern edge of the Tibetan Plateau. They found that model structural uncertainty exceeds meteorological input uncertainty. Similar conclusions emerge from~\cite{kennedy2025JAMES}, who examined a parameter-perturbed ensemble and reported that model parameter uncertainty can surpass meteorological input uncertainty. The substantial uncertainties inherent in model structures and parameters render the distinction of different atmospheric models' outputs with streamflow observations in mountainous regions an ill-posed problem~\cite{renard2010WRR}: a given atmospheric model output corresponds to a wide range of feasible streamflow simulations due to hydrological model uncertainties, while multiple atmospheric models can produce similar streamflow outputs if their uncertainty ranges overlap. When streamflow observations fall within these overlapping uncertainty ranges, distinguishing between atmospheric models becomes infeasible. This challenge parallels that encountered when using precipitation for atmospheric model evaluation over the Tibetan Plateau~\cite{collier2024CD}.

Several studies have attempted to mitigate hydrological model uncertainties through parameter calibration. However, we contend that such efforts are largely ineffective for atmospheric model evaluation in mountainous regions where hydrological model structural uncertainty is substantial. While calibration can reduce parameter uncertainty, it fails to address structural uncertainty~\cite{clark2011WRR, clark2016WRR}. More significantly, when structural uncertainty is prominent, calibration can render hydrological model parameters adaptive to model inputs, producing similar streamflow simulations despite different meteorological inputs. \citeA{chen2025W} compared multiple publicly available streamflow reanalyses for the Upper Jinsha River basin, revealing that despite substantial differences in meteorological inputs and hydrological models~\cite{alfieri2020JHX, harrigan2020ESSD, yang2021BAMS}, the reanalyses yield similar streamflow estimates due to extensive calibration, reasonably reproducing observed streamflow. These results prove that streamflow observations alone are insufficient to distinguish between different meteorological inputs to hydrological models following calibration in mountainous regions.

Recognizing that the challenge primarily arises from hydrological model structural uncertainty, several studies have sought to reduce this uncertainty. We propose that these efforts can be classified into two principal approaches. The first approach leverages additional observations to impose stronger constraints on model structure. \citeA{henn2015WRR, henn2016WRR} employed snow water equivalent observations to constrain model structural uncertainties and developed a Bayesian inference framework for estimating basin-mean precipitation from streamflow observations. Their results demonstrated that structural constraints render year-to-year variations in inferred precipitation robust, enabling evaluation of long-term atmospheric model simulations~\cite{rudisill2023GMD}. The second approach utilizes simpler models with fewer parameters to reduce uncertainty. \citeA{adam2006JC} applied the Budyko framework---a simple hydrological model---to infer long-term basin-averaged precipitation from streamflow observations. This framework characterizes the relationship between long-term averaged precipitation and runoff using minimal parameters, thereby reducing uncertainty relative to complex hydrological models. \citeA{barkhordari2025JH, wang2019CSB} enhanced this method by integrating remote-sensing evapotranspiration data into the Budyko framework, further reducing reliance on calibratable parameters. However, both approaches are limited to evaluating long-term basin-averaged precipitation, and their reliability for short-term variations (within a year), non-precipitation meteorological variables, and scenarios with unclosed water budget observations~\cite{zheng2020JAMES, tan2022WRR} remains largely unverified.

Hydrological models typically represent two fundamental processes: runoff generation and flow routing. Although runoff generation processes exhibit considerable complexity in mountainous regions~\cite{van_tiel2024NW}, flow routing is generally less intricate than in flat terrain~\cite{getirana2013WRR, moussa1996JH}. Mountain rivers and terrain, characterized by steep slopes, facilitate water flow representation primarily through the kinematic wave approximation of the Saint--Venant equations~\cite{moussa1996JH} (channel routing) or the shallow water equations (terrain routing). This representation demonstrates that water flow dynamics are predominantly governed by the interaction between friction and terrain slope~\cite{getirana2013WRR, moussa1996JH}. Since terrain slopes can be derived globally from high-resolution digital elevation models~\cite{yamazaki2017GRL, yamazaki2019WRR}, water flow routing in mountainous basins depends substantially on friction assumptions. Friction represents a relatively minor component in comprehensive hydrological models. Characterizing friction uncertainties is typically more straightforward than addressing the structural and parameter uncertainties inherent in hydrological models. The relative simplicity of routing processes in mountainous basins may result in lower uncertainty in streamflow simulations. Consequently, flow routing models may be more suitable for evaluating atmospheric models in these regions.

Building upon this rationale, we propose an alternative approach for evaluating atmospheric models that exclusively utilizes flow network routing. This methodology directly routes atmospheric model-simulated runoff through the flow network to generate streamflow estimates at gauge locations. We hypothesize that due to the relative simplicity of routing processes in mountainous basins, the uncertainty contributed by the routing model is small enough to allow statistically significant discrimination among different atmospheric model configurations.

This study aims at demonstrate that a parsimonious network routing model can serve as a robust evaluation tool for kilometer-scale atmospheric simulations in data-scarce mountain regions. We applied the proposed method to the Yarlung Zangbo River basin. The basin is the largest river basin on the Tibetan Plateau and is now the site of the world's largest hydropower development project. Accurate atmospheric simulations are essential for operating and managing hydropower facilities along the river. Yet their evaluation is hindered by sparse in-situ observations and large uncertainties in satellite retrievals. The river extends approximately 2,000 kilometers, with an average elevation exceeding 4,000 meters above sea level. Its catchment area spans approximately 2.5 million square kilometers. The basin's scale precludes direct runoff--streamflow comparison at sub-monthly time scales, while its steep, high-altitude terrain introduces large structural uncertainty in hydrological models. These characteristics make it an ideal testbed for routing network-based evaluation method.

The structure of this paper is as follows: Section~\ref{sec:methods} delineates the methodologies and datasets employed in our study. Section~\ref{sec:results} presents the analytical outcomes. Finally, Section~\ref{sec:conclusions} synthesizes the principal findings and provides concluding remarks.

\section{Methods and Data}\label{sec:methods}

\subsection{Experimental Design}

We implemented the proposed methodology to evaluate different parameterization schemes of the Weather Research and Forecasting (WRF) model~\cite{powers2017BAMS}. Figure~\ref{fig:workflow} illustrates the experimental workflow. A series of thirteen WRF experiments with 3-kilometer grid spacing were conducted from May 1 to October 1, 2013, encompassing the wet season. We selected 2013 because its streamflow closely approximates long-term climatology, and the wet season accounts for the majority of annual streamflow~\cite{zhou2021CD}. The initial 50 days served as a spin-up period (May 1 to June 19), sufficiently long to mitigate the effects of potentially inaccurate initial conditions, including snowpack. Simulations from June 20 to October 1 were analyzed.

The WRF experiments were initialized and driven by data from the European Centre for Medium-Range Weather Forecasts Reanalysis version 5 (ERA5)~\cite{hersbach2020QJRMS}. WRF-simulated runoff was spatially re-mapped to represent lateral inflow into a pre-delineated flow network. Water flow within this network was subsequently computed using the Muskingum method, a well-established approach for routing mountainous river flows with minimal calibratable parameters.

Routing parameters and associated uncertainties were estimated through a three-step procedure. First, optimal parameter values were determined for each WRF experiment by calibrating against streamflow observations. Calibration followed an upstream-to-downstream sequence, ensuring that routing parameters for upstream gauges were established prior to those for downstream gauges. This sequential approach captures spatial dependencies of routing parameters along the river network. Second, optimal parameter values were used to fit log-normal distributions at each river gauge. Third, these fitted distributions were subsequently used to generate ensembles of 100 random parameter sets. Flow routing ensembles were executed using these parameter sets for each WRF experiment, enabling estimation of routing uncertainty and statistical assessments.

The WRF experiments were intercompared according to their skill in reproducing observed streamflow. Statistical significance of performance differences between them was assessed. Statistically significant disparities thereby support the study’s central hypothesis.

\subsection{Observational Data}

Streamflow data were obtained from the China Three Gorges Corporation and Ministry of Water Resources of China, collected at six river gauges: Lazi, Nugesha, Lhasa, Yangcun, Gengzhang, and Nuxia. Figure~\ref{fig:domain} illustrates the spatial distribution of these gauge locations. Lazi, Nugesha, Yangcun, and Nuxia are situated along the mainstream in upstream-to-downstream sequence. Lhasa is located on a tributary between Nugesha and Yangcun, while Gengzhang lies between Yangcun and Nuxia. Instantaneous streamflow observations were recorded hourly from June 20 to October 1, 2013.

To contextualise the proposed evaluation method, we examined how tightly the evaluation results track independent evaluations of the same WRF experiments. We used the Global Precipitation Measurement (GPM) Multi-satellitE Retrievals for GPM (IMERG) product~\cite{huffman2019GPM} as an illustrative benchmark. The GPM IMERG product provides data at 0.1° spatial resolution and 30-minute temporal resolution. The data were bilinearly interpolated to the 3-kilometer WRF grid to permit pixel-wise comparison.

\begin{figure*}[h!]
  \centering
  \noindent\includegraphics[width=140mm]{workflow.pdf}
  \caption{Schematic diagram of the workflow of this study. The parameterization schemes used in the WRF experiments are listed in Table~\ref{tab:wrf_experiment}.}\label{fig:workflow}
\end{figure*}

\begin{figure*}[h!]
  \centering
  \noindent\includegraphics[width=140mm]{domain.pdf}
  \caption{WRF domain, boundary of the Yarlung Zangbo River, and the delineated flow network. The colormap provides a representation of the terrain elevation within the WRF domain, showcasing the basin's topographical characteristics. Black lines denote the basin boundary. Red dots indicate the locations of the four river gauges along the river's course, labeled as follows: LZ for Lazi, NGS for Nugesha, LS for Lhasa, YC for Yangcun, GZ for Gengzhang, and NX for Nuxia. Colored lines correspond to the flow paths that lie between consecutive gauges, offering a visual guide to their spatial distribution across the basin. The inset illustrates the study domain, depicted as a rectangle, along with the coastlines, which are represented by dashed lines.}\label{fig:domain}

\end{figure*}

\subsection{Study Area and Flow Routing Network}

Figure~\ref{fig:domain} displays the WRF model domain, which spans 380 by 660 grid cells, each measuring 3 by 3 kilometers. The domain covers the entire Yarlung Zangbo River basin and extends to include a buffer zone surrounding the basin. This buffer zone, which is over 200 kilometers wide, is strategically designed to allow the development of small-scale weather systems before they interact with the river basin~\cite{denis2002CD}. This design helps to mitigate adverse effects stemming from inaccurate or low-resolution boundary conditions.

The routing was performed on the network as illustrated in Figure~\ref{fig:domain}. The routing network was delineated from the Multi-Error Removed Improved-Terrain Hydrography (MERIT-Hydro) dataset~\cite{yamazaki2017GRL, yamazaki2019WRR}, which provides flow directions and accumulative upstream area data at 3 by 3 arcseconds. The delineation process proceeds in three sequential steps: Initially, the accumulative upstream area is employed to identify flow paths, with a grid cell being classified as such if its accumulative upstream area exceeds 10 km\textsuperscript{2}. Subsequently, these flow paths are segmented from upstream to downstream, defining a flow path segment by an increase in the accumulative upstream area of at least 20 km\textsuperscript{2}. Finally, the flow direction data are utilized to determine the catchment area of each flow path segment. The thresholds for defining flow paths aligns with that used in previous large-domain river routing studies~\cite{lin2021SD, lin2019WRR}. This delineation process results in a fully connected network consisting of 5,800 flow path segments, with an average catchment area of 33~km\textsuperscript{2}.

\subsection{WRF Parameterization Schemes}\label{sec:wrf_experiment}

Table~\ref{tab:wrf_experiment} presents the parameterization schemes selected from WRF version 4.3.3. Thirteen experiments were conducted, a number that satisfies the minimum requirement for statistical estimation of celerity distribution while maintaining computational feasibility. Experiments were configured by systematically modifying one parameterization scheme at a time while maintaining consistency across other schemes. This design minimizes inter-experiment differences, thereby increasing the challenge of statistical significance tests and enhancing the robustness of study findings.

The experiments are organized into four groups based on the process being modified: radiation, cloud microphysics, planetary boundary layer, and orographic drag. These parameterizations have demonstrated significance in previous investigations~\cite{lv2020ESS, prein2023CD}.

The first group comprises five experiments (E01 to E05) examining radiation parameterization impacts. E03 was configured similarly to the High Asian Refined Analysis version 2~\cite{wang2021IJC}, with modifications to radiation and land surface processes. The Rapid Radiative Transfer Model for GCMs (RRTMG)~\cite{iacono2008JGRA} replaced the Rapid Radiative Transfer Model (RRTM) scheme~\cite{mlawer1997JGRA} for both shortwave and longwave radiation transfer. RRTMG provides comparable radiative forcing modeling to RRTM while offering enhanced computational efficiency~\cite{iacono2008JGRA}. For land surface processes, the Noah land surface model with multiparameterization options (Noah-MP)~\cite{niu2011JGRA,yang2011JGRA} was selected instead of the Noah model used in HARR version 2. Noah-MP incorporates improvements over the original Noah model, particularly in representing snow and runoff processes~\cite{niu2011JGRA}. These enhancements have yielded superior runoff modeling performance~\cite{liang2019AAS, zheng2023ESSD}, leading to Noah-MP's widespread adoption in hydrological applications~\cite{cosgrove2024JAWRA, lin2018JHM}. Among Noah-MP's various runoff generation parameterization options, this study employed the Noah runoff scheme, which utilizes exponentially distributed infiltration capacity for runoff generation~\cite{schaake1996JGRA} and assumes free drainage at the soil column base to simulate subsurface runoff. Experiments E01 to E05 systematically replaced shortwave and longwave radiation schemes individually, with the widely used Dudhia and Goddard schemes selected for these perturbations~\cite{dudhia1989JAS, matsui2020CD}.

The second group includes three experiments (E03, E06, and E07) investigating cloud microphysics scheme effects. E03 employs the Thompson scheme~\cite{thompson2008MWR}, while E06 and E07 utilize the Purdue Lin scheme~\cite{chen2002JMSJ} and the WRF Single-Moment 6-Class Microphysics (WSM6) scheme~\cite{hong2006APJAS}, respectively. These three schemes represent the most commonly used single-moment microphysics parameterizations in WRF applications~\cite{hong2006APJAS, thompson2008MWR, chen2002JMSJ}.

The third group contains five experiments (E03, E08 to E12) examining planetary boundary layer scheme impacts. Given the tight coupling between surface layer and shallow convection schemes with planetary boundary layer schemes, these were also examined within this group. E03 uses the Mellor--Yamada--Janji'c scheme~\cite{janjic1994MWR}, while E08 to E12 employ: the Mellor--Yamada--Nakanishi--Niino level 2.5 scheme~\cite{nakanishi2006BM, nakanishi2009JMSJ}, the Yonsei University scheme~\cite{hong2006MWR}, the Asymmetric Convective Model version 2 (ACM2) scheme~\cite{pleim2007JAMC, pleim2007JAMCa}, the Quasi-Normal Scale Elimination (QNSE) scheme~\cite{sukoriansky2005BM}, and the Yonsei University scheme coupled with the Global/Regional Integrated Modeling System shallow convection scheme~\cite{hong2018APJAS}. These schemes are extensively used in WRF applications and have proven effective for simulating boundary layer processes across diverse climatic regions.

The fourth group encompasses two experiments (E08 and E13) exploring orographic drag parameterization effects. Building upon E08, E13 incorporates small-scale~\cite{tsiringakis2017QJRMS} and turbulent orographic drag~\cite{beljaars2004QJRMS}.

\begin{table}[h!]
  \doublerulesep 0.3pt
  \footnotesize
  % \tabcolsep 7.8mm
  % \setlength{\tabcolsep}{15pt}   %%%设置表格平均分配列间距
  \renewcommand{\arraystretch}{1}  %%%设置表格行高
  \caption{WRF experiments and corresponding parameterization schemes. Radiation column shows shortwave and longwave schemes, respectively. Planetary boundary layer column indicates planetary boundary layer, surface layer, and shallow convection parameterizations, respectively. For orographic drag, L and S denote large- and small-scale gravity waves; B and T denote flow blocking and turbulent orographic form drag, respectively. Parameterization abbreviations and full references are provided in Section~\ref{sec:wrf_experiment}.}\label{tab:wrf_experiment}
  \vspace*{5mm}
  \begin{tabular*}{140mm}{ccccc}
    \hline
    Experiment & Radiation & Cloud Microphysics & Planetary Boundary Layer & Orographic Drag \\
    \hline
    E01 & Dudhia \& RRTMG & Thompson & MYJ \& Eta \& - & LB \\
    E02 & Dudhia \& RRTMG & Thompson & MYJ \& Eta \& - & LB \\
    E03 & RRTMG \& RRTMG & Thompson & MYJ \& Eta \& - & LB \\
    E04 & RRTMG \& Goddard & Thompson & MYJ \& Eta \& - & LB\\
    E05 & Goddard \& Goddard & Thompson & MYJ \& Eta \& - & LB \\
    E06 & RRTMG \& RRTMG & Prudue Lin & MYJ \& Eta \& - & LB \\
    E07 & RRTMG \& RRTMG & WSM6 & MYJ \& Eta \& - & LB \\
    E08 & RRTMG \& RRTMG & Thompson & MYNN2 \& MYNN \& EDMF & LB \\
    E09 & RRTMG \& RRTMG & Thompson & YSU \& MM5 \& - & LB \\
    E10 & RRTMG \& RRTMG & Thompson & ACM2 \& MM5 \& - & LB \\
    E11 & RRTMG \& RRTMG & Thompson & QNSE \& QNSE \& QNSE & LB \\
    E12 & RRTMG \& RRTMG & Thompson & YSU \& MM5 \& GRIMS & LB \\
    E13 & RRTMG \& RRTMG & Thompson & MYNN2 \& MYNN \& EDMF & LBST \\
    \hline
  \end{tabular*}
  \renewcommand{\arraystretch}{1}  %%%设置表格行高
\end{table}

\subsection{Flow Routing Method}

The WRF model includes a routing module known as WRF-Hydro~\cite{lin2018EMS, givati2016H}. WRF-Hydro comprehensively integrates subsurface runoff, soil moisture redistribution, diffusive terrain routing, and Muskingum--Cunge channel routing. However, its complexity introduces a large number of parameterization options and parameters, making thorough calibration and uncertainty quantification challenging~\cite{rafieeinasab2025WRR}, particularly in data-sparse regions such as the Yarlung Zangbo River basin~\cite{lei2025JH}.

As described in the Introduction, routing processes in mountainous regions can be substantially simplified. In alignment with our study purpose, we opted for Muskingum routing over a dense flow network instead of WRF-Hydro. A dense flow routing network functions not only as a prescribed water flow path for channel routing but also, to some extent, as a surrogate for terrain routing. When sufficiently dense, terrain routing within small catchments with steep slopes can be neglected.

The WRF-simulated runoff was re-mapped to represent lateral inflow into the flow routing network, following the remapping method detailed in~\cite{lin2018EMS, wang2019CSB}. Unlike \citeA{lin2018EMS}, we identified all WRF grid cells that intersect with the catchment of a water flow path segment, rather than using a single grid cell at the centroid of the catchment. Runoff volume from these grid cells was then calculated by multiplying the runoff depth by the intersection area for each cell. These individual values were subsequently summed to determine the total lateral flow volume entering the flow path segment. This method conserves the total runoff volume despite variations in network geometry.

The Muskingum method is well-suited to characterize flow's kinematic wave propagation driven by the topographic gradient~\cite{ponce1978JHD}. The formulation of the Muskingum method~\cite{cunge1969JHD, fenton2019JH}, when incorporating lateral inflows, can be expressed as follows:
\begin{eqnarray}
  Q_{i}^{t} = \frac{k - x}{1 - x + k} Q_{i-1}^{t} + \frac{1 - x - k}{1 - x + k} Q_{i}^{t-1} + \frac{x + k}{1 - x + k} Q_{i-1}^{t-1} + \frac{2k}{1 - x + k} Q_l^t \textrm{,} \label{eq:muskingum}\\
  k  = \frac{c \Delta t} {2 \Delta l} \textrm{,}
\end{eqnarray}
where $Q_{i}^{t}$ ($\textrm{m}^3\,\textrm{s}^{-1}$) is the streamflow at current time step, $Q_{i-1}^{t}$ ($\textrm{m}^3\,\textrm{s}^{-1}$) is the streamflow at the upstream position at current time step, $Q_{i}^{t-1}$ ($\textrm{m}^3\,\textrm{s}^{-1}$) is the streamflow at previous time step, $Q_{i-1}^{t-1}$ ($\textrm{m}^3\,\textrm{s}^{-1}$) is the streamflow at the upstream position at the previous time step, $Q_l^t$ ($\textrm{m}^3\,\textrm{s}^{-1}$) is the lateral inflow at current time step. $\Delta t$ is the time step ($\textrm{s}$), and $\Delta l$ is the flow path length ($\textrm{m}$). $x$ is the weighting factor (unitless), and $c$ is the wave celerity ($\textrm{m}\,\textrm{s}^{-1}$).

The application of the Muskingum method to a fully connected flow network must be executed in the correct order. Streamflow at the upstream position must be available before routing. We sorted all flow paths within the network according to stream order as introduced in~\cite{yang2024W}. This sorting ensures upstream flow paths are always prioritized over their downstream counterparts. We then apply the Muskingum method to each flow path segment in the established sequence. This sequence guarantees that routing occurs first on upstream segments, followed by downstream segments.

The Muskingum method introduces two adjustable routing parameters: the weighting factor ($x$; unitless) and wave celerity ($c$; $\textrm{m}\,\textrm{s}^{-1}$). The weighting factor $x$ regulates the proportional contributions of the right-hand-side terms in Equation~\ref{eq:muskingum}. Previous studies demonstrate that the simulated streamflow is relatively insensitive to variations in the weighting factor~\cite{koussis1978JHD}. Values ranging from 0.1 to 0.3 typically prove effective for most streams. Guided by the experiments reported by \citeA{david2011JHM}, we have chosen a parameter value of 0.3 for this study.

The wave celerity $c$ represents the speed at which the flow wave propagates downstream. This parameter is influenced by the physical characteristics of the flow path, including slope, roughness, and width. Given that direct observations of $c$ are unavailable for the Yarlung Zangbo River basin, we calibrated $c$ using streamflow observations. The calibration is performed for each river gauge sequentially from upstream to downstream. For each gauge, a set of optimal celerity values are obtained by driving the routing model with WRF-simulated runoff. The set of optimal celerity values is then fitted to a log-normal distribution. This fitted distribution aims to capture the variations in wave celerity calibration arising from different routing model inputs.

\subsection{Evaluation Metrics}

We utilized the Pearson correlation coefficient to identify the optimal wave celerity value. The Pearson correlation coefficient ($r$) is defined as follows:
\begin{equation}
  r = \frac{\sum_{i=1}^{n} (x_i - \bar{x})(y_i - \bar{y})}{\sqrt{\sum_{i=1}^{n} (x_i - \bar{x})^2 \sum_{i=1}^{n} (y_i - \bar{y})^2}} \textrm{,}
\end{equation}
where $x_i$ and $y_i$ are the simulated and observed streamflow at time step $i$, respectively. $\bar{x}$ and $\bar{y}$ are the mean of the simulated and observed streamflow, respectively. The Pearson correlation coefficient ranges from \textminus{}1 to 1, with a value of 1 indicating a perfect positive linear relationship between the simulated and observed streamflow. This coefficient is insensitive to biases in the streamflow estimates, making us to focus on daily scale streamflow variations in this study.

We used the Kling--Gupta efficiency (KGE)~\cite{gupta2009JH} to intercompare the WRF parameterization schemes. KGE sythematically summarizes how a hydrological simulation matches observations in correlation, standard deviation, and bias. The KGE is defined as follows:
\begin{eqnarray}
  \textrm{KGE} = 1 - \sqrt{\left(r - 1\right)^2 + \left(\alpha  - 1\right)^2 + \left(\beta - 1\right)^2} \textrm{,} \\
  \alpha  = \frac{\sigma}{\sigma_o} \textrm{,}                                                                 \\
  \beta = \frac{\mu_s}{\mu_o} \textrm{,}
\end{eqnarray}
where $r$ is the Pearson correlation coefficient between the simulation and observation, $\sigma$ is the standard deviation of the simulation, and $\mu$ is the mean. The subscripts $s$ and $o$ denote the simulated and observed values, respectively. The KGE ranges from $-\infty$ to 1. A KGE value of 1 indicates a perfect match between the simulated and observed streamflow.

\section{Results and Dicussion}\label{sec:results}

Our analysis commences with intercomparing runoff simulated by the WRF experiments. We then estimate routing parameters and their associated uncertainties. Finally, we evaluate WRF experiment performance in streamflow estimation and assess whether the experiments can be distinguished given uncertainty introduced by the routing process.

Substantial differences in WRF-simulated runoff are evident across the thirteen experiments (Figure~\ref{fig:runoff_coefficient_of_variation}). The inter-experiment spread, quantified by standard deviation, generally exceeds the climatological mean, particularly in the middle and upper reaches of the Yarlung Zangbo River. The runoff spread substantially exceeds that observed for precipitation (Figure~S1). Figure~\ref{fig:runoff_coefficient_of_variation}b suggests that runoff variations primarily correspond to radiation parameterization differences, a pattern not evident for precipitation. Comparison between accumulated runoff and snow water equivalent (Figure~S2) reveals negligible snow water equivalent relative to runoff, suggesting that runoff spread is not attributable to radiation-modulated snow accumulation and ablation processes. The spread in precipitation minus evapotranspiration (Figure~S3) exhibits spatial patterns and magnitudes similar to runoff, indicating that runoff spread likely originates from covariation between precipitation and evapotranspiration. Given that radiation parameterization significantly influences surface temperature~\cite{liu2023AR, lv2020ESS} while precipitation shows marginal yet detectable responses~\cite{hui2019AR}, radiation emerges as a substantial source of uncertainty in runoff simulation. Beyond radiation, parameterizations of cloud microphysics, planetary boundary layer, and small-scale orographic drag also contribute to runoff variability (Figure~\ref{fig:runoff_coefficient_of_variation}c--e).

\begin{figure*}[h!]
  \centering
  \noindent\includegraphics[width=140mm]{runoff_coeffcient_of_variation.pdf}
  \caption{Coefficient of variation of runoff averaged from June 20 to October 1, 2013. (a) Standard deviation of runoff from 12 WRF simulations (excluding E10) divided by mean runoff. (b) Standard deviation from the radiation group (E01--E05) divided by the mean runoff from the 12 WRF experiments. (c) Standard deviation from the cloud microphysics group (E03, E06, and E07) divided by mean runoff. (d) Standard deviation from the planetary boundary layer group (E03, E08, E09, E11, and E12) divided by mean runoff. (e) Standard deviation from E08 and E13 divided by mean runoff, representing orographic drag uncertainty.}\label{fig:runoff_coefficient_of_variation}

\end{figure*}

Despite substantial differences in simulated runoff, optimal celerity values yielding the highest correlation coefficients exhibit consistency across WRF experiments at each gauge (Figure~\ref{fig:celerity_estimation}). The median optimal correlation coefficient across examined gauges and WRF experiments is 0.75, demonstrating that the Muskingum routing model effectively captures water wave propagation dynamics. The Muskingum method is simple. As hypothesized, this simple method demonstrates robustness, rendering celerity calibration insensitive to model input variations. The consistent optimal celerity values among the WRF experiments reflect intrinsic flow path characteristics in mountainous basins. Specifically, optimal celerity is generally slower in upstream regions compared to downstream areas. As illustrated in Figure~\ref{fig:domain}, downstream regions at the Tibetan Plateau margin feature broader river channels and steeper slopes, corresponding to elevated celerity values.

Optimal celerity values from WRF experiments were fitted to log-normal distributions, with expectations and standard deviations presented in Figure~\ref{fig:celerity_estimation}. Experiment E10 was excluded from this fitting procedure, as it consistently yielded correlation coefficients lower than other experiments (Figure~\ref{fig:celerity_estimation}). Given that this underperformance is statistically significant (Figure~S4), the exclusion does not compromise assessment of flow routing uncertainty or the validity of the routing-based evaluation method. The log-normal distributions generated 100 sets of random celerity values. Ensemble streamflow simulations using these celerity value sets were conducted to estimate uncertainty associated with the routing process.

\begin{figure*}[h!]
  \centering
  \noindent\includegraphics[width=140mm]{celerity_estimation.pdf}
  \caption{Celerity calibration at river gauges: (a) Lazi, (b) Nugesha, (c) Lhasa, (d) Yangcun, (e) Gengzhang, and (f) Nuxia. Colored lines show correlation coefficient variations with celerity. Black plus signs indicate optimal celerity values yielding highest correlation coefficients for each WRF experiment. Optimal values are fitted to log-normal distributions. Vertical black lines and gray zones denote expectation and standard deviation of log-transformed optimal celerity values, with corresponding values indicated.}\label{fig:celerity_estimation}

\end{figure*}

Figure~\ref{fig:streamflow_simulation} depicts ensemble means and uncertainty ranges of streamflow simulated for each WRF experiment at the six river gauges. The WRF experiments exhibit substantial variability in streamflow estimates, reflecting pronounced differences in simulated runoff. Ensemble spread associated with the routing process is relatively small compared to inter-experiment differences. This routing uncertainty is also substantially smaller than uncertainty associated with hydrological process parameterization estimated in~\cite{lei2025JH}. Visual inspection indicates that simulations from E01 to E07 consistently overestimate streamflow, with overestimation magnitudes significantly exceeding the ensemble range. In contrast, experiments E08, E09, E11, and E12 produce visually similar streamflow estimates across all gauges. These four experiments more closely reproduce observed streamflow. We proceed to examine how flow routing-based evaluation distinguish different WRF experiments using quantitative performance metrics.

\begin{figure*}[h!]
  \centering
  \noindent\includegraphics[width=140mm]{streamflow_simulation.pdf}
  \caption{Ensemble mean and uncertainty range of simulated streamflow at (a) Lazi, (b) Nugesha, (c) Lhasa, (d) Yangcun, (e) Gengzhang, and (f) Nuxia. Solid lines represent ensemble averages for each WRF experiment, with shaded areas showing routing process uncertainty. Black dots indicate observations.}\label{fig:streamflow_simulation}

\end{figure*}

Figure~\ref{fig:streamflow_kge} intercompares the WRF experiments by their median KGE. The most skillful experiments appear leftmost. The interquartile range of the KGE values from the ensemble simulations are displayed for comparison. The WRF experiment performance varies with location. No single experiment consistently outperforms all others at all gauges. However, when considering frequency of ranking among the top three performers across the six gauges, E11 demonstrates slight superiority, achieving this position five times. E09 follows, being the top three four times. E12 ranks third with three top-three placements. Experiment rankings exhibit minor variations depending on the performance metric employed. Figure~S5 presents results using correlation coefficient as the evaluation metric. E12 emerges as the most skillful experiment with five top-three rankings, while E09 and E11 both achieve three top-three rankings.

\begin{figure*}[h!]
  \centering
  \noindent\includegraphics[width=140mm]{streamflow_kge.pdf}
  \caption{Kling--Gupta Efficiency (KGE) at gauges: (a) Lazi, (b) Nugesha, (c) Lhasa, (d) Yangcun, (e) Gengzhang, and (f) Nuxia. Black boxes show interquartile ranges of KGE values across ensemble simulations. Horizontal lines within boxes denote median KGE values. Black dots indicate outliers. Experiments are sorted by median KGE values, with leftmost experiments being most skillful.}\label{fig:streamflow_kge}

\end{figure*}

Statistical significance of performance differences between all WRF experiment pairs was assessed using ensemble KGE values at a 0.05 significance level. Figure~\ref{fig:kge_significance} displays results for KGE, while Figure~S6 presents corresponding analyses for correlation coefficient. Both analyses demonstrate that differences among WRF experiments remain statistically significant relative to routing process uncertainty. This finding indicates that the routing model exhibits sufficiently low uncertainty to distinguish performance among different WRF configurations, thereby supporting the hypothesis articulated in the Introduction.

\begin{figure*}[h!]
  \centering
  \noindent\includegraphics[width=140mm]{kge_difference_significance.pdf}
  \caption{Statistical significance of Kling-Gupta Efficiency (KGE) differences across WRF experiments at gauges: (a) Lazi, (b) Nugesha, (c) Lhasa, (d) Yangcun, (e) Gengzhang, and (f) Nuxia. Differences between experiment pairs are tested using paired t-tests. Gray color indicates significance at 0.05 level; white indicates non-significant differences.}\label{fig:kge_significance}

\end{figure*}

The procedure of fitting optimal celerity values to log-normal distributions requires a number of WRF experiments, which is often computationally impractical given the substantial demands of kilometer-scale atmospheric simulations. An alternative approach utilizes calibrated celerity values for individual WRF experiments without distribution fitting, though this simplified method does not incorporate routing process uncertainty. We examined whether this alternative approach yields rankings comparable to the uncertainty-incorporating methodology. Figure~\ref{fig:kge_rank} compares WRF experiment rankings based on median KGE values from ensemble simulations with rankings derived from KGE values using calibrated celerity for each experiment. Table~\ref{tab:kge_rank} presents Spearman correlation coefficients between these ranking approaches. Rankings exhibit strong consistency, with Spearman correlation coefficients exceeding 0.99 across all gauges. These findings demonstrate that calibrated celerity values provide reliable experiment rankings even without explicit uncertainty quantification. This consistency remains robust across various performance metrics. Comparable consistency is observed for correlation coefficient rankings, with Spearman correlation coefficients surpassing 0.90 at all river gauges (Figure~S1 and Figure~S8). This high consistency arises from routing uncertainty being relatively minor compared to inter-experiment performance differences, as illustrated in Figure~\ref{fig:streamflow_kge}.

Figure~\ref{fig:kge_rank} and Table~\ref{tab:kge_rank} further compare WRF experiment rankings based on KGE of basin-averaged precipitation and runoff with rankings derived from streamflow KGE. Basin-averaged precipitation KGE is calculated against the GPM IMERG precipitation product, while runoff KGE is computed by directly comparing area-weighted sums of WRF-simulated runoff with observed streamflow, thereby neglecting routing processes. These approaches represent commonly employed evaluation methods for atmospheric models.

For runoff evaluation, Table~\ref{tab:kge_rank} hints that correspondence exists between higher rankings based on streamflow KGE and higher rankings based on runoff KGE. However, this correspondence lacks robustness across river gauges. Detailed analysis reveals that correspondence between these evaluation methods varies considerably with gauge location. At the upstream Lazi gauge, significant streamflow overestimation (Figure~\ref{fig:streamflow_simulation}) results in bias dominating KGE contributions. High bias in streamflow correlates strongly with high bias in runoff, producing robust correspondence between runoff KGE and streamflow KGE rankings. At Nugesha, Lhasa, Yangcun, and Nuxia, correspondence primarily emerges from low-performing WRF experiments (ranked beyond position 7 in Figure~\ref{fig:kge_rank}), which exhibit substantial biases. For high-performing experiments, correspondence remains weak. At Gengzhang, all WRF experiments demonstrate similar bias magnitudes (Figure~\ref{fig:streamflow_simulation}), resulting in consistently weak correspondence between runoff KGE and streamflow KGE rankings. If the bias is excluded from the skill measure, the correspondence is consistently weak, as shown for correlation coefficient in Figure~S7. Critically, visual comparison between runoff and streamflow time series indicates that neglecting routing processes is problematic at hourly time scales (Figure~S8), as they exhibit substantial differences. Based on these findings, we conclude that runoff-based evaluation methods neglecting routing processes lack reliability for atmospheric model evaluation at hourly time scales.

For precipitation evaluation, correspondence with streamflow KGE-based rankings generally mirrors the pattern observed for runoff. We contend that this similarity stems from the fundamental relationship where higher precipitation typically generates higher runoff under consistent land surface conditions. These findings suggest that when atmospheric models exhibit substantial precipitation biases, basin-averaged precipitation evaluation can provide indicative insights for streamflow assessment. However, when atmospheric models demonstrate sound precipitation simulation in basin-averaged temporal means, flow routing-based evaluation becomes essential for hydrologically relevant assessment. The two evaluation approaches may yield divergent performance rankings of atmospheric models. This discrepancy reflects the nonlinear response of streamflow to the spatiotemporal distribution of precipitation,\cite{stephens2015GRL} and the influence of non-precipitation variables (e.g., the contrasting variability of runoff and precipitation in Figure~\ref{fig:runoff_coefficient_of_variation} and Figure~S1). Consequently, precipitation- and routing-based evaluations are complementary rather than redundant.

\begin{figure*}[h!]
  \centering
  \noindent\includegraphics[width=140mm]{rank_kge_relationship.pdf}
  \caption{Relationship between the rank of the median Kling--Gupta Efficiency (KGE) measured by streamflow and the rank measured by other observations across the WRF experiments at the gauge of (a) Lazi, (b) Nugesha, (c) Lhasa, (d) Yangcun, (e) Gengzhang, and (f) Nuxia. Dashed lines represent 1:1 lines indicating perfect agreement. Dots show optimal KGE ranks, plus signs indicate runoff-based ranks, and triangles denote basin-averaged precipitation ranks.}\label{fig:kge_rank}
\end{figure*}

\begin{table}[h!]
  \centering
  \doublerulesep 0.3pt
  % \tabcolsep 7.8mm
  % \setlength{\tabcolsep}{15pt}
  \renewcommand{\arraystretch}{1}
  \caption{Spearman's correlation coefficient between the rank of the median Kling--Gupta Efficiency measured by streamflow and the rank measured by other observations or skill measures across the WRF experiments.}\label{tab:kge_rank}
  \vspace*{5mm}
  \begin{tabular*}{90mm}{cccc}
    \hline
    Gauge & Calibration & Runoff & Precipitation \\
    \hline
    Lazi & 0.99 & 0.98 & 0.90 \\
    Nugesha & 0.99 & 0.88 & 0.87 \\
    Lhasa & 1.00 & 0.84 & 0.84 \\
    Yangcun & 1.00 & 0.87 & 0.86 \\
    Gengzhang & 0.99 & -0.49 & -0.87 \\
    Nuxia & 0.99 & 0.69 & 0.63 \\
    \hline
  \end{tabular*}
  \renewcommand{\arraystretch}{1}
\end{table}

\section{Conclusions}\label{sec:conclusions}

This study develops and validates a routing network-based methodology for evaluating atmospheric models using streamflow observations in data-sparse mountainous regions. The proposed approach distinguishes itself from conventional hydrological model-based evaluations by requiring substantially fewer assumptions regarding model structures and parameters. We hypothesize that by propagating runoff through a parsimonious network routing model, the additional uncertainty is kept small enough to discriminate among atmospheric model configurations.

The application of this methodology to thirteen WRF experiments in the Yarlung Zangbo River basin with varying parameterization schemes demonstrates its effectiveness in distinguishing atmospheric model performance. The Muskingum routing model proves robust and effective, with optimal celerity values exhibiting consistency across different WRF configurations and producing median correlation coefficients of 0.75 across gauges for hourly streamflow. Statistical significance testing confirms that routing model uncertainty remains sufficiently low to distinguish performance among different WRF configurations, supporting our central hypothesis.

The routing-based evaluation provides complementary insights to precipitation assessment using the GPM IMERG product. Specifically, when atmospheric models demonstrate sound precipitation simulation in basin-averaged temporal means, routing-based evaluation becomes essential for hydrologically relevant atmospheric model assessment. This approach captures aspects of model performance that precipitation-based evaluation alone may overlook.

The methodology's practical implementation is facilitated by the finding that calibrated celerity values provide reliable experiment rankings even without explicit uncertainty quantification, with Spearman correlation coefficients exceeding 0.99 between uncertainty-incorporating and simplified approaches. This computational efficiency makes the method particularly valuable given the substantial demands of kilometer-scale atmospheric simulations.

\section*{Conflict of Interest}

The authors declare no conflicts of interest relevant to this study.

\section*{Open Research Section}

The GPM IMERG final run precipitation product version 7 was used to evaluate the WRF-simulated precipitation~\cite{huffman2019GPM}. The ERA5 reanalysis data~\cite{hersbach2020QJRMS} were used to drive the WRF model version 4.3.3~\cite{powers2017BAMS} to simulate precipitation and runoff. The MERIT-Hydro flow direction and cumulative upstream area data~\cite{yamazaki2019WRR} were used to delineate the routing network. The delineated routing network, WRF-simulated precipitation and runoff, and the code for routing and parameter optimization are available at \url{https://doi.org/10.57760/sciencedb.11618}~\cite{zheng2024SCB}. The streamflow observations for the Yarlung Zangbo River were obtained from China Three Gorges Corporation and Ministry of Water Resources of China; however, they are not shareable due to licensing restrictions.

\acknowledgments{}

This study is supported by the National Key Research and Development Program of China (2023YFF0805501), China Yangtze Power Co., Ltd. (contract Z532302035), and the Natural Science Foundation of China (grants 42075165 and 42275178).

\bibliography{references}

\end{document}
